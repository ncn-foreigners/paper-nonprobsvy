\documentclass[
]{jss}

%% recommended packages
\usepackage{orcidlink,thumbpdf,lmodern}

\usepackage[utf8]{inputenc}

\author{
Łukasz Chrostowski\\Adam Mickiewicz University \And Piotr
Chlebicki~\orcidlink{0009-0006-4867-7434}\\Stockholm University
\AND Maciej Beręsewicz~\orcidlink{0000-0002-8281-4301}\\Poznań
University of Economics and Business
}
\title{\pkg{nonprobsvy} -- An R package for modern methods for
non-probability surveys}

\Plainauthor{Łukasz Chrostowski, Piotr Chlebicki, Maciej Beręsewicz}
\Plaintitle{nonprobsvy -- An R package for modern methods for
non-probability surveys}
\Shorttitle{\pkg{nonprobsvy} for non-probability surveys}


\Abstract{
The abstract of the article.
}

\Keywords{keywords, not capitalized, \proglang{R}}
\Plainkeywords{keywords, not capitalized, R}

%% publication information
%% \Volume{50}
%% \Issue{9}
%% \Month{June}
%% \Year{2012}
%% \Submitdate{}
%% \Acceptdate{2012-06-04}

\Address{
    Łukasz Chrostowski\\
    Pearson\\
    First line\\
Second line\\
  E-mail: \email{lukchr@st.amu.edu.pl}\\
  URL: \url{https://posit.co}\\~\\
      Piotr Chlebicki\\
    Stockholm University\\
    Department of Statistics and Mathematics,\\
Faculty of Biosciences,\\
Universitat Autònoma de Barcelona\\
  E-mail: \email{piotr.chlebicki@math.su.se}\\
  
      Maciej Beręsewicz\\
    Statistical Office in Poznań\\
    Department of Statistics,\\
Institute of Informatics and Electronic Economy,\\
Poznań University of Economics and Business\\
  E-mail: \email{maciej.beresewicz@ue.poznan.pl}\\
  URL: \url{https://posit.co}\\~\\
  }


% tightlist command for lists without linebreak
\providecommand{\tightlist}{%
  \setlength{\itemsep}{0pt}\setlength{\parskip}{0pt}}




\usepackage{amsmath} \newcommand{\bX}{\boldsymbol{X}} \newcommand{\bx}{\boldsymbol{x}} \newcommand{\bY}{\boldsymbol{Y}} \newcommand{\by}{\boldsymbol{y}} \newcommand{\bh}{\boldsymbol{h}} \newcommand{\bH}{\boldsymbol{H}} \newcommand{\ba}{\boldsymbol{a}} \newcommand{\bp}{\boldsymbol{p}} \newcommand{\bA}{\boldsymbol{A}} \newcommand{\bw}{\boldsymbol{w}} \newcommand{\bd}{\boldsymbol{d}} \newcommand{\bZ}{\boldsymbol{Z}} \newcommand{\bz}{\boldsymbol{z}} \newcommand{\bv}{\boldsymbol{v}} \newcommand{\bu}{\boldsymbol{u}} \newcommand{\bU}{\boldsymbol{U}} \newcommand{\bQ}{\boldsymbol{Q}} \newcommand{\bG}{\boldsymbol{G}} \newcommand{\HT}{\text{\rm HT}} \newcommand{\bbeta}{\boldsymbol{\beta}} \newcommand{\balpha}{\boldsymbol{\alpha}} \newcommand{\btau}{\boldsymbol{\tau}} \newcommand{\bgamma}{\boldsymbol{\gamma}} \newcommand{\btheta}{\boldsymbol{\theta}} \newcommand{\blambda}{\boldsymbol{\lambda}} \newcommand{\bPhi}{\boldsymbol{\Phi}} \newcommand{\bEta}{\boldsymbol{\eta}} \newcommand{\bZero}{\boldsymbol{0}} \newcommand{\colvec}{\operatorname{colvec}} \newcommand{\logit}{\operatorname{logit}} \newcommand{\Exp}{\operatorname{Exp}} \newcommand{\Ber}{\operatorname{Bernoulli}} \newcommand{\Uni}{\operatorname{Uniform}}

\begin{document}



\section{Introduction}\label{introduction}

This template demonstrates some of the basic LaTeX that you need to know
to create a JSS article.

\subsection{Code formatting}\label{code-formatting}

In general, don't use Markdown, but use the more precise LaTeX commands
instead:

\begin{itemize}
\item
  \proglang{Java}
\item
  \pkg{plyr}
\end{itemize}

One exception is inline code, which can be written inside a pair of
backticks (i.e., using the Markdown syntax).

If you want to use LaTeX commands in headers, you need to provide a
\texttt{short-title} attribute. You can also provide a custom identifier
if necessary. See the header of Section \ref{r-code} for example.

\section[R code]{Methods for non-probability samples \proglang{R}
code}\label{r-code}

\subsection{Basic setup}\label{basic-setup}

Let \(U=\{1,..., N\}\) denote the target population consisting of \(N\)
labelled units. Each unit \(i\) has an associated vector of auxiliary
variables \(\boldsymbol{x}_{i}\) (a realisation of the random vector
\(\boldsymbol{X}_{i}\) in the super-population) and the study variable
\(y_{i}\) (a realisation of the random variable \(Y_{i}\) in the
super-population). Let \(\{ (y_i, \boldsymbol{x}_i), i \in S_A\}\) be a
dataset of a non-probability sample of size \(n_A\) and let
\(\{\left(\boldsymbol{x}_i, \pi_{i}\right), i \in S_B\}\) be a dataset
of a probability sample of size \(n_B\), where only information about
variables \(\boldsymbol{X}\) and inclusion probabilities \(\pi\) (which
in the super population model are also considered to be random
variables) are available. Let \(\delta\) be an indicator of inclusion
into non-probability sample. Each unit in the sample \(S_B\) has been
assigned a\textasciitilde design-based weight given by
\(d_i = 1/\pi_i\). The setting is summarised in Table ..

The goal is to estimate a\textasciitilde finite population mean
\(\displaystyle\mu_{y}=\frac{1}{N}\sum_{i=1}^{N} y_{i}\) of the target
variable \(Y\). As values of \(y_{i}\) are not observed in the
probability sample, it cannot be used to estimate the target quantity.
Instead, one could try combining the non-probability and probability
samples to estimate \(\mu_{y}\). In this paper we do not consider
modifications for the possibly occurring overlap.

\subsection{Mass Imputation
estimators}\label{mass-imputation-estimators}

\subsection{Inverse Probability Weighting
estimators}\label{inverse-probability-weighting-estimators}

\subsection{Doubly Robust estimators}\label{doubly-robust-estimators}

Can be inserted in regular R markdown blocks.

\begin{CodeChunk}
\begin{CodeInput}
R> x <- 1:10
R> x
\end{CodeInput}
\begin{CodeOutput}
 [1]  1  2  3  4  5  6  7  8  9 10
\end{CodeOutput}
\end{CodeChunk}

\subsection[Features specific to rticles]{Features specific to
\pkg{rticles}}\label{features-specific-to}

\begin{itemize}
\tightlist
\item
  Adding short titles to section headers is a feature specific to
  \pkg{rticles} (implemented via a Pandoc Lua filter). This feature is
  currently not supported by Pandoc and we will update this template if
  \href{https://github.com/jgm/pandoc/issues/4409}{it is officially
  supported in the future}.
\item
  Using the \texttt{\textbackslash{}AND} syntax in the \texttt{author}
  field to add authors on a new line. This is a specific to the
  \texttt{rticles::jss\_article} format.
\end{itemize}

\section{Package contents and
implementation}\label{package-contents-and-implementation}

\section{Practical examples}\label{practical-examples}

\renewcommand\refname{Summary}
\bibliography{references.bib}



\end{document}
